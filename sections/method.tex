\section{方法}
\subsection{编程与写作环境}
本文采用C/C++编译器为GCC 5.1.1,系统环境为Fedora 23,代码编辑器为VIM 7.4,项目编译工具为Make 4.0,文档编辑工具为Latex。
\subsection{贪心算法求最小生成树}
贪心算法一般不能解决实际问题,但是只要能使用贪心算法解决的问题,那么对于该问题而言,此算法即为最优的。
\subsection{开源支持}
\subsubsection{股票数据调取项目——Tushare}
本文中所有股票数据全部来自于Tushare财经接口工具包项目,该项目采用Python编码,从各大官方财经站点调取股票数据。

\section{概念}
\subsection{研究背景与意义}

\subsubsection{研究背景}
对于风格投资的研究,国外的已有文献较多,Farrell\cite{Farrell}和Sharpe\cite{Farrell}都发现了股票收益间较高的相关性,与风格相关。一些实证研究也表明,基础价值协同性并不是收益协同性的唯一根源,Fama\cite{Farrell}和French的研究表明某些股票的收益协同性与基础价值协同性无关;另外,大量文献证明了初基础价值外,投资者偏好、交易地点、现金流、指数都可以导致收益协同性形成风格。Barberis等总结了已有的研究成果并提出了风格投资协同理论——风格投资产生收益协同性。对于同种风格的股票之间存在较强的相关性。

\subsubsection{研究作用与意义}
指数分层结构图横坐标为股票名称,纵坐标为股间距离。通过绘制指数分层结构图,可以清晰的看到股票的聚集状态,推导出风格的分布和组合样本。继而为研究风格形成因素、收益及风险水平等性质开辟了道路,为证券投资组合配置提供依据和参考。从上述可知,指数分层结构图对风格投资分析的重要性,然而在绘制指数分层结构图时用到的数据相当庞大,因此研究指数分层结构图的绘制算法具有实际的意义与价值。

